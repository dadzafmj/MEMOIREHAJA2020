%%%%%%%%%%%%%%%%%%%%%%%%%%%%%%%%%%%%%%%%%%%%%
%		LISTE DES ACRONYMES
%%%%%%%%%%%%%%%%%%%%%%%%%%%%%%%%%%%%%%%%%%%%%
\newacronym{espa}{ESPA}{Ecole Supérieure Polytechnique d'Antsiranana}
\newacronym{saas}{SaaS}{Software as a Service}
\newacronym{sla}{SLA}{Service-Level Agreement}
\newacronym{soa}{SOA}{Service Oriented Architecture}
\newacronym{api}{API}{Interface de Programmation Applicative}
\newacronym{god}{GoD}{Gaming on Demand}
\newacronym{lmd}{LMD}{Licence-Master-Doctorat}
\newacronym{ue}{UE}{Unité d'Enseignement}
\newacronym{ec}{EC}{Elément Constitutif}
\newacronym{sis}{SIS}{Student Information Systems}
\newacronym{crm}{CRM}{Customer Relationship Management}
\newacronym{grc}{GRC}{Gestion de la Relation Client}
\newacronym{sgbd}{SGBD}{Système de Gestion de Base de Données}
\newacronym{jee}{JEE}{Java Etreprise Edition}
\newacronym{html}{HTML}{Hypertext Markup Language}
\newacronym{css}{CSS}{Cascading Style Sheets}
\newacronym{mvc}{MVC}{Model View Controller}
\newacronym{jsf}{JSF}{Java Server Faces}
\newacronym{jsp}{JSP}{Java Server Pages}
\newacronym{jstl}{JSTL}{JavaServer Pages Standard Tag Library}
\newacronym{rh}{RH}{Ressources Humaines}
\newacronym{grh}{GRH}{Gestion des Ressources Humaines}
\newacronym{bdd}{BDD}{Base de Données}
\newacronym{uml}{UML}{Unified Modeling Language}
\newacronym{cv}{CV}{Curriculum Vitæ}
\newacronym{mcd}{MCD}{Modèle Conceptuel des Données}
\newacronym{mld}{MLD}{Modèle Logique de Données}

\newacronym{json}{JSON}{JavaScript Object Notation}
\newacronym{spa}{SPA}{Single-Page Application}
\newacronym{adsl}{ADSL}{Asymmetric Digital Subscriber Line}



%%%%%%%%%%%%%%%%%%%%%%%%%%%%%%%%%%%%%%%%%%%%%
%		LISTE DES GLOSSAIRES
%%%%%%%%%%%%%%%%%%%%%%%%%%%%%%%%%%%%%%%%%%%%%
\newglossaryentry{multiTenant}{
	type=notation,
	name={Multi-tenant},
	description={En informatique, multi-tenant, ou multi-entité désigne un principe d'architecture logicielle permettant à un logiciel de servir plusieurs organisations clientes à partir d'une seule installation},
}
\newglossaryentry{cloudComputing}{
	type=notation,
	name={Cloud computing},
	description={Le cloud computing, ou l'informatique en nuage ou nuagique ou encore l'infonuagique, est l'exploitation de la puissance de calcul ou de stockage de serveurs informatiques distants par l'intermédiaire d'un réseau, généralement internet},
	sort={N}%
}
\newglossaryentry{gamingOnDemand}{
	type=notation,
	name={Gaming on Demand},
	description={Jeu à la demande en français, est une technologie similaire à la vidéo à la demande permettant de jouer normalement à des jeux vidéo sur son écran d'ordinateur ou sa télévision alors que celui-ci tourne sur des serveurs à distance qui renvoient la vidéo de ce qui a été joué en lecture en continu (communément appelé streaming)},
	sort={N}%
}
\newglossaryentry{middleware}{
	type=notation,
	name={Middleware},
	description={Ou intergiciel est un logiciel tiers qui crée un réseau d'échange d'informations entre différentes applications informatiques.},
	sort={N}%
}