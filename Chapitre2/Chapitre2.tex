%%%%%%%%%%%%%%%%%%%%%%%%%%%%%%%%%%%%%%%%%%%%%%%%%%%%%%%%%%%%%%%%%%%%%%%%%%%%%%%%%%%%%%%%%%%%%
%%									   Chapitre 2	    								 %%
%%%%%%%%%%%%%%%%%%%%%%%%%%%%%%%%%%%%%%%%%%%%%%%%%%%%%%%%%%%%%%%%%%%%%%%%%%%%%%%%%%%%%%%%%%%%%
\chapter{Nature et définition des services fournis}
\minitoc
\newpage
%%%%%%%%%%%%%%%%%%%%%%%%%%%%%%%%%%%%%
\section{Quand les ambitions deviennent réalité}
Dans le secteur de l'éducation supérieur, devenu de plus en plus concurrentiel,
le succès des organisations/établissements repose sur quatre ambitions
fondamentales : stimuler la réussite des étudiants, soutenir la croissance de l'établissement, améliorer l'efficacité administrative et alléger les efforts de gestion de personnel.
\medskip

Le succès dans chacun de ces domaines est important mais les institutions doivent exceller dans chacun d'eux. Une telle réussite dépend de solutions technologiques modernes et complètes, capables d'intégrer immédiatement, améliorations, innovations et ruptures technologiques.
\medskip

La solution Grou4Success permet de répondre aux ambitions des établissent de manière spécifique et pour toute leur organisation :

\begin{itemize}
	\item Mise à jour aisée et rapide des stratégies et procédures sans intervention technique coûteuse.
	\item Implémentations plus efficaces, permettant de dédier les ressources à l'excellence institutionnelle plutôt qu'à la maintenance du système.
	\item Expérience utilisateur accrue, combinée avec des outils de support libérant les équipes technique et administratives pour les tâches à plus forte valeur ajoutée.
	\item Disponible partout et tout le temps, sur PC, Mac et tablette.
	\item Un tableau de bord orienté rôles qui affiche les chiffres clés et une vue d'ensemble simple et directe des anomalies.
\end{itemize}

\section{Ambition : alléger les efforts de gestion de personnel}

Au sein d'une entreprise, la gestion du personnel demande la prise en charge, au quotidien, de l'ensemble des tâches administratives liées au personnel, de la paie des salariés, du développement et de tout ce qui est lié aux ressources humaines. Pour soutenir la réussite dans ces efforts, trois domaines d'action sont à privilégier.

\subsection{Mieux connaître et contrôler les absences}
Outre les congés légaux ou conventionnels, rémunérés ou non, qui sont généralement prévisibles et donc plus faciles à anticiper, un certain nombre d'événements peuvent générer une absence plus ou moins longue du salarié.
\medskip

La plus délicate d'entre elles est certainement l'absence injustifiée. Il peut s'agir d'une absence sans autorisation préalable de l'employeur, ou consécutive à son refus d'autorisation, ou d'une prolongation sans autorisation d'une absence initialement autorisée.
\medskip

En cas de maladie, il doit lui transmettre un certificat médical, dans un délai fixé par convention collective, règlement intérieur ou usage, et est le plus souvent de 48 heures. Il s'agit là du délai admis pour justifier l'absence, ce qui n'empêche pas le salarié à en informer l'employeur beaucoup plus rapidement.

\subsection{Maîtriser les emplois du temps}
Il apparaît aujourd'hui que le temps est devenu une variables stratégique pour les entreprises. Bien gérer le temps de travail permet de gagner en efficacité et en productivité. Le temps est donc une ressources rare et non renouvelable. Avec l'entreprise, l'expression « \textit{le temps, c'est de l'argent} » prend alors toute son importance. Le temps doit donc être géré, contrôlé, planifié. 

\subsection{Organiser la gestion de la paie}
La gestion de la paie constitue un aspect important de la gestion des personnels. Matérialisant la relation entre employeur et le salarié, le bulletin de paie est un document périodique obligatoire devant répondre à certaines règles. Le salaire, contrepartie de la prestation du travail effectuée par le salarié, constitue un coût pour l'entreprise qui doit être comptabilisé. 
\medskip

Les heures complémentaires sont les heures que l'employeur peut demander au salarié à temps partiel d'effectuer au-delà de la durée fixée par le contrat de travail, ils doivent être évaluées afin de calculer le montant de la rémunération de l'employé.

\subsection{Group4Success apporte une réponse}
Group4Success fournit des outils de gestion du personnel qui gère toute l'administration des ressources humaines d'une entreprise, c'est à dire que c'est lui qui s'occupe des recrutements, contrats, fiches de paie, dossiers des salariés, et autres documents règlementaires comme les cotisations sociales par exemple. Sa tâche est aussi de vérifier la bonne adaptation des salariées à leur milieu professionnel, contrôler leurs savoirs et compétences sur le terrain, s'assurer du respects des règles et des horaires.
\medskip

L'automatisation des procédures d'absence accélère et simplifie grandement l'enregistrement et la gestion des absences, générant par conséquent un fort gain d'efficacité. D'un côté, les utilisateurs reçoivent des alertes proactives pour garder la trace des tâches qui leur sont assignés; quant au système, il crée la représentation la plus claire possible des absences en rassemblant tous les types de demandes d'absences et leurs mises à jour, dans un dossier unique et complet pour chaque employé.
\medskip

A chaque salarié concerné le contrôle de son emploi du temps, avec droits de modification sur son calendrier personnel qui affiche une vue d'ensemble des tâches prévues. Il peut ainsi enregistrer lui-même ses demandes de congés, d'absence et les demandes d'avance ou encore mettre à jour ses activités planifiées ou y apporter des changements.
\medskip

\subsection{Déduction des tâches-clés pour satisfaire l'ambition}
\subsubsection{Données générales}
\begin{itemize}
	\item[\textbullet] Création des fiches employeur : nom, adresse, contact, salaire, date d'embauche, heure de travail.
	\item[\textbullet] Paramètres légaux et conventionnels.
	\item[\textbullet] La gestion des contrats de prévoyance.
\end{itemize}

\subsubsection{Gestionnaire RH}
\begin{itemize}
	\item[\textbullet] La gestion des emplois.
	\item[\textbullet] La gestion de paie : journal de paie, formule de calcul, prime d'ancienneté, 13\up{ème} mois et les avances. 
	\item[\textbullet] Traitement de masse : augmentation générales, recrutement, évaluation des entretiens d'embauches, contrat d'embauche, la gestion des requête pour approuver les demandes d'absences et congés.
	\item[\textbullet] Visualisation, analyses, éditions, contrôles des personnelles et exportation des données.
	\item[\textbullet] Suivi des évènement : visites médicales, affectation de matériels.
\end{itemize}

\subsubsection{Personnelles}
\begin{itemize}
	\item[\textbullet] Demande d'absence.
	\item[\textbullet] Demande de congés.
	\item[\textbullet] Demande d'avance.
\end{itemize}

\section{Ambition : soutenir la croissance de l'établissement}
De nombreuses écoles de l'enseignement supérieur ainsi que des universités
font face à la dure réalité de la diminution de leurs ressources, menaçant leur
existence même. La situation n'est certes pas aussi sombre pour toutes, mais ne
pas anticiper un risque de perte de revenus peut mettre en péril la survie d'une institution. Pour la plupart, cela signifie qu'il faut pouvoir tout à la fois, maintenir ou augmenter le nombre d'inscriptions, offrir de nouveaux programmes flexibles à destination d'une population croissante d'apprenants "non traditionnels", ou encore lever des fonds auprès de donateurs ou d'anciens élèves. Chaque méthode apporte ses propres défis à surmonter \cite{unit4}.

\subsection{Augmenter le nombre d'inscriptions}
La baisse de nombre d'inscriptions d'étudiants a un impact financier sérieux sur de nombreuses institutions de l'enseignement supérieur. Pour y faire face, elles ne peuvent que se lancer dans une compétition effrénée.
\medskip

Dans le même temps, les étudiants eux-mêmes sont devenus plus avertis de ces pratiques pour les recruter. Pour peser sur leur choix, toute stratégie de marketing doit s'appuyer sur des données provenant de systèmes de gestion de la relation clients performants. Ceci implique souvent l'utilisation de systèmes tiers dans l'écosystème de l'institution, induisant un surcoût.

\subsection{Répondre aux besoins d'apprenants}
L'époque d'une seule et même stratégie d'enseignement destinées à de jeunes adultes étudiant à temps plein est révolue : il s'agit maintenant de répondre aux besoins d'élèves souvent plus âgés qui suivent des cursus à temps partiel.
\medskip

Les systèmes d'information destinés aux étudiants conçus avant ce nouveau phénomène d'apprentissage non-traditionnel ne sont pas prévus pour répondre à ces exigences d'organisation très spécifiques, en termes de recrutement ou de programmes. De la même façon, les applications financières ne sont plus adaptées pour gérer la profitabilité des offres d'enseignement traditionnel ou non traditionnel.

\subsection{Améliorer la collecte de fonds}
L'utilisation de procédés dépassés influe considérablement sur l'efficacité de la création et de la gestion des campagnes de collecte de fonds : augmenter le niveau des donations devient alors encore plus difficile.
\medskip

De la même façon, le fait de considérer donateurs et anciens élèves séparément sans tenir compte de leur historique avec l'institution restreint les possibilités de rapprochements utiles, qui pourtant pourraient êtes source de message ciblés impactant.

\subsection{Group4Success apporte une réponse}
Group4 Success Management fournit les outils nécessaires pour interagir efficacement dans la durées avec chaque partie prenante : personnels, étudiants, anciens élèves, partenaires extérieurs et mécènes.
\medskip

Les workflows configurables via un simple « drag \& drop » permettant d'automatiser les campagnes de sensibilisation en fonction du comportement des prospects, des anciens élèves, ou des donateurs. Le même \glsentryshort{crm} (\glsentrylong{crm}) étant disponible pour tout utilisateur de l'institution, nous l'appelons le « CRM pour tous », qui donne une vue à 360° sur chaque individu durant sa progression en tant que candidat potentiel, étudiant, diplômé, ancien élève. Instituant ainsi une relation qui peut durer plusieurs décennies.
\medskip

Group4 Success Management répond également aux besoins des apprenants non traditionnels d'aujourd'hui : sa totale flexibilité dans la gestion des inscriptions, sans contrainte de cours ou de programmes, lève la barrière de la procédure classique de recrutement trimestriel.
\medskip

L'application permet aussi de faire des analyses exhaustives sur l'ensemble des données, nous pouvons déterminer la profitabilité de chaque programme ou département et prendre les décisions relatives à l'expansion des programmes d'enseignement.

\subsection{Les tâches-clés pour satisfaire l'ambition}
Des modules spécifiques sont disponibles pour gérer les tâches-clés pour satisfaire l'ambition. D'après ce qu'on a mentionné ci-dessus, la procédure se déroule comme suit : 

\begin{itemize}
	\item Améliorer la gestion des inscription en établissant des critères propres selon les besoins de l'établissement.
	\item Définir des campagnes de sensibilisation pour mieux avertir les étudiants au recrutements.
	\item Faire des suivis de l'institution.
	\item Offrir des nouveaux programme flexibles.
	\item Obtenir une vue globale des anciens élèves et donateurs pour tirer profit de le leurs relations avec les départements et les groupes auxquels ils appartiennent.
	\item Lancer des actions permettant de générer des revenus complémentaires comme la gestion des locaux, le logement ou les programmes de formation continue.
\end{itemize}

\section{Ambition : stimuler la réussite des étudiants}
Les institutions s'investissent spontanément dans la réussite de leurs étudiants et les stimulent pour réussir et en contre partie, les risques consécutifs à l'échec n'ont jamais été aussi forts. Le choix des étudiants pour une école dépend aussi du soutien que l'on va lui proposer, de la valeur et des opportunités d'employabilité de l'institution. Cette pression est également accentuées par les régulateurs, les organismes d'accréditation et les sources gouvernementales de financement, qui tiennent les institutions garantes du progrès des étudiants.
\medskip

Trois domaines clés doivent être considérés pour amener l'ensemble des étudiants à la réussite attendue.

\subsection{Remédier efficacement au risque d'échec}
L'identification et l'intervention en cas de risque d'échec forment une composante fondamentale de notre stratégie de réussite étudiante. Comment identifier précisément les individus ayant la plus forte probabilité d'échec académique ?
\medskip

Déterminer les catégories d'étudiants en risque d'échec est un premier pas, mais le réel tournant est de pouvoir identifier ces personnes. Il est fort probable que notre système d'information étudiant ne soit pas doté d'une telle capacité en amont.

\subsection{Clarifier les exigences de niveau}
Les étudiants sont censés achever leur cursus dans un laps de temps spécifique. Qu'il s'agisse de deux ans, quatre ans, ou de toute autre durée, ne pas en respecter le terme entraîne des frais d'études supplémentaires et retarde le bénéfice de l'obtention du diplôme.
\medskip

Il est par conséquent regrettable que tant d'étudiants fassent des choix inadaptés, faute de savoir précisément à quoi s'attendre lors du parcours retenu. Cela résulte fatalement en un allongement de la durée d'obtention des diplômes ou, bien pire, en l'abandon pur et simple des études.

\subsection{Améliorer l'expérience étudiant}
Les systèmes d'information étudiants ont eu jusqu'à présent une fonction secondaire mais les institutions souhaitent offrir toute une palette de services utiles aux étudiants et au personnel.
\medskip

Le milieu étudiant (et professoral) est aujourd'hui familier des nombreuses applications commerciales d'utilisation intuitive et accessibles sur tout type de terminal. L'image de l'institution pâtit forcément d'un système d'information étudiant obsolète ou compliqué, et la productivité de ses utilisateur s'en trouve amoindrie.

\subsection{Group4Success apporte une réponse}
La technologie est maintenant au centre de l'expérience des étudiants : ce sont des utilisateur « clients » et ils sortent du lycée en espérant être accueillis dans des institutions équipées de solutions de dernière génération. Qu'il s'agisse d'emploi du temps, de facturation, ou de paiement des cours, ils exigent des solution du niveau des applications qu'ils utilisent déjà dans la vie de de tous les jours et qu'ils n'hésitent pas à partager avec le reste du monde sur les réseaux sociaux, en donnant leurs impressions sur le sujet. Pour ces institutions, le défi consiste donc à pouvoir offrir à leurs étudiants les applications interactives et intuitives qu'ils attendent ainsi que la puissance de travail et d'analyse dont leurs personnels ont besoin.
\medskip

Afin de prévenir l'échec des étudiants, Group4 Success Management fournit des outils d'évaluation de risque permettant d'intervenir à temps auprès des étudiants concernés. En créant un profil de risque unique pour la population étudiante de l'institution, pour pouvoir interagir avec les personnes en difficulté au travers de plans individualisés tout en bénéficiant d'une collaboration inter-départements.
\medskip

Group4 Success Management aide également les étudiants à rester dans la course grâce à un outil complet d'évaluation du parcours diplômant. Cet outil permet aux étudiants de consulter les pré-requis d'obtention de semestre, année ou diplôme, leur permettant ainsi de rester motivés et bien informés pour être prêts à temps. Mieux encore, les étudiants peuvent effectuer des simulations pour évaluer leurs progrès et modifier éventuellement leur spécialité au cours de leurs études.
\medskip

Le système LMD favorise l'autonomie et le travail personnel de l'étudiant. Un accompagnement numérique doit être proposé aux étudiants, comprenant notamment l'accès à des bibliothèques de ressources universitaires en ligne et la connexion à un « Environnement Numérique de Travail ». Grâce à une interface intuitive bâtie pour les utilisateurs, Group4 Success Management offre un haut niveau dans la facilité d'utilisation et la fiabilité des fonctionnalités, pour que les étudiants disposent de tout ce dont ils ont besoin sur leur propre matériel. Une grande qualité de service, toujours le « CRM pour tous » qui donne accès aux outils de la relation clients à l'ensemble des départements.

\subsection{Synthèse des tâches-clés pour satisfaire l'ambition}
Une description des tâches détermine les responsabilités et les fonctionnalités exigées afin de répondre aux besoins de l'application. Ces tâches comprend les informations suivantes :

\begin{itemize}
	\item La création d'un profil de risque unique, pour remédier au risque d'échec des étudiants.
	\item Recevoir les notifications et fournir l'outil pour l'évaluation de risque.
	\item Fournir un outil d'évaluation du parcours, permettant au étudiant de consulter les pré-requis d'obtention de semestre, année ou diplôme.
	\item Fournir un outil pour visualiser la suivi des étudiants.
	\item Offrir aux étudiants des applications interactives et intuitives.
	\item Combler leurs attentes technophiles avec une application disponible tout le temps, partout et sur n'importe quel périphérique.
	\item Architecture adaptative et configurable.
\end{itemize}
 
\section{Ambition : améliorer l'efficacité administrative}
La pression sur les coûts éducatifs, les restrictions budgétaires et les attentes toujours plus élevées des étudiants incitent à un nouveau modèle économique pour les écoles supérieures et les universités : il leur faut impérativement trouver de nouvelles voies dans la gestion des ressources. La gestion des coûts et l'amélioration de la productivité sont devenues de nouvelles normes de fonctionnement.
\medskip

Pour permettre à l'institution d'améliorer son efficacité administrative, trois domaines requièrent plus particulièrement notre attention.

\subsection{Automatiser les procédures courantes}
Les procédures manuelles constituent une perte de temps et d'argent significative. Elles retardent le service aux étudiants, favorisent les erreurs d'enregistrement de données et peuvent générer une non-conformité importante aux règles administratives.
\medskip

Les procédures opérées manuellement persistent toutefois, généralement du fait de systèmes dépassés et incapables de prendre en charge de nouveaux modes de fonctionnement, ou encore à cause de systèmes administratifs variés présentant le risque de duplication de saisie.

\subsection{Soutenir la prise de décision sur données}
Systèmes dérivés et opération verticalisées rendent les données moins exploitables : un agglomérat d'informations opaques décourage la collaboration, menace l'intégrité des données et peut rendre impossible l'analyse de mesures de performance.
\medskip

Dans l'impossibilité de récupérer, tester et visualiser la masse de données que vous détenez, tout effort de collecte et de stockage est privé d'intérêt. Votre capacité d'amélioration s'en trouve fortement amoindrie, et l'impact sur votre image peut s'avérer énorme.

\subsection{Optimiser les ressources humaines et financières}
Sans système de gestion administrative unifié, vous éprouvez les pires difficultés à établir des budget proactifs et à contrôler vos dépenses, bataillant avec des sources de données varié pour comparer la profitabilité de vos programmes et départements.
\medskip

Il est difficile, en outre, de quantifier le retour sur investissement du capital humain et d'identifier les meilleurs talents, surtout si vous utilisez des systèmes RH (Ressources Humaines) et paie séparés, qui complexifient encore davantage la prise en compte des évolutions réglementaires.

\subsection{Group4Success apporte une réponse}
La plateforme Group4Success apporte une technologie à la pointe de la modernité visant à libérer les intervenants afin qu'ils puissent se concentrer sur leur cœur de métier, et appliquant les principes des applications grand public que nous avons l'habitude d'utiliser quotidiennement. Le résultat en est une expérience utilisateur accrue, qui limite au maximum les nombreuses tâches fastidieuses sans valeur ajoutée et permet ainsi à l'utilisateur d'être plus présent et plus efficace là où il le choisit.
\medskip

Group4Success automatise le cycle de vie de l'étudiant et élimine les dépenses liées à la maintenance de systèmes agglomérés. Les données intégrées sont en effet partagées entre les différentes fonctions, permettant un accès instantané à l'information pour toutes les parties prenantes et éliminant les doublons, notamment lors de la saisie. Il est aussi plus rapide et plus facile à déployer, vous pouvez donc immédiatement profiter de votre investissement. 
\medskip

Le plateforme regroupe aussi la finance, les ressources humaines, la gestion des projets et les achats, offre une efficacité et des bénéfices tangibles. Vous pouvez notamment améliorer votre transparence financière en accédant et en analysant les données de chaque département, individuellement ou sur l'ensemble de votre organisation. Le déroulement optimisé des processus et les outils intégrés améliorent notablement la productivité aidant à réduire les coûts, et augmentent vos revenus en évaluant au mieux les opportunités qui se présentent.

\subsection{Aperçu des tâches-clés pour satisfaire l'ambition}
Nous retiendrons que pour améliorer l'efficacité administratif, des tâches spécifiques sont à mettre en place : 

\begin{itemize}
	\item Automatisation des procédures courantes : certificat de scolarité, carte étudiant et bibliothèque.
	\item Construction des rapports détaillables par l'outil de tableau de bord permettant de zoomer jusqu'au détail et d'utiliser toutes les informations de votre données.
	\item Consultation, visualisation et analyse des données par campus, département, programme.
	\item Accès instantané à l'information pour tous les parties prenantes.
\end{itemize}
