%%%%%%%%%%%%%%%%%%%%%%%%%%%%%%%%%%%%%%%%%%%%%%%%%%%%%%%%%%%%%%%%%%%%%%%%%%%%%%%%%%%%%%%%%%%%%
%%									   Remerciement	    								 %%
%%%%%%%%%%%%%%%%%%%%%%%%%%%%%%%%%%%%%%%%%%%%%%%%%%%%%%%%%%%%%%%%%%%%%%%%%%%%%%%%%%%%%%%%%%%%%
\chapter*{Remerciements}
\addstarredchapter{Remerciements}

%%%%%%%%%%%%%%%%%%%%%%%%%%%%%%%%%%%%%
%%  redéfinition de l'interligne   %%
%%%%%%%%%%%%%%%%%%%%%%%%%%%%%%%%%%%%%

{\setlength{\baselineskip}{1.5\baselineskip}

%%%%%%%%%%%%%%%%%%%%%%%%%%%%%%%%%%%%%

En préambule de ce travail, j'aimerai exprimer, par ces quelques lignes de remerciements,
mes gratitudes envers tous ceux en qui, par leur présence, leur soutien, leur disponibilité
et leurs conseils j'ai trouvé courage pour réaliser ce travail.
\bigskip

En premier lieu, j'adresse mes respectueuses considérations à mon encadreur
professionnel : Dr. RAKOTOARISOA Jean Claude,Monsieur RAMANAN'HAJA Hery Tina qui ont été l'esprit pensant donnant
naissance à ce sujet et qui m'a dirigée par ses précieux conseils tout en me laissant la
liberté d'initiative pour la conception de la plateforme. A mes encadreurs qui sont Dr Luigi BELLINI,Madame	RAZAIARIMALALA Nirina Claudia
pour avoir dirigés ce travail, pour leurs soucient et aussi pour leurs aides et leurs
orientations qui m'ont permis de réaliser ce travail dans les meilleures conditions.
\bigskip

J'adresse mes remerciements anticipés et mes honorables considérations également aux
membres de Jury qui vont évaluer et porter leur jugement à ce travail, aussi en enrichir le
contenu par leurs précieuses propositions.
\bigskip

Je dois reconnaissance également à l'ensemble du corps enseignant de la mention STIC
de le ESPA (Ecole Supérieure Polytechnique d'Antsiranana).
\bigskip

Une mention particulière à ma famille, pour leurs soutiens et leurs attentions sans faille,
dont les encouragements et l'amour.
\bigskip

Bref, tout ceux qui ont contribué de près ou de loin à l'accomplissement de ce travail.

\null\hfill HAJALALAINA Fara Marie José

%%%%%%%%%%%%%%%%%%%%%%%%%%%%%%%%%%%%%%
%% fin redéfinition de l'interligne %%
%%%%%%%%%%%%%%%%%%%%%%%%%%%%%%%%%%%%%%

\par}

%%%%%%%%%%%%%%%%%%%%%%%%%%%%%%%%%%%%%

