%%%%%%%%%%%%%%%%%%%%%%%%%%%%%%%%%%%%%%%%%%%%%%%%%%%%%%%%%%%%%%%%%%%%%%%%%%%%%%%%%%%%%%%%%%%%%
%%									   Conclusion  	    							  	   %%
%%%%%%%%%%%%%%%%%%%%%%%%%%%%%%%%%%%%%%%%%%%%%%%%%%%%%%%%%%%%%%%%%%%%%%%%%%%%%%%%%%%%%%%%%%%%%
\chapter*{Conclusion générale}
\addstarredchapter{Conclusion générale}

Notre projet intitulé Développement de SaaS « Group4SuccessManagement » est destiné
pour alléger les efforts de gestion de personnel, soutenir la croissance de l’établissement,
stimuler la réussite des étudiants et améliorer l'efficacité administrative.
\medskip

Contrairement à la majorité des applications classique, notre application de type SaaS
fournit des services aux utilisateurs en échange d'abonnement mensuels, elle est installés
sur des serveurs distants plutôt que sur la machine de l'utilisateur. En utilisant ce service
l’établissement client n'hébergera pas l'application et ne stockera pas ses données en
interne. Les utilisateurs devront simplement disposer d’un dispositif muni d'un navigateur web et des codes d’accès au service en ligne pour pouvoir travailler.
\medskip

En ce qui concerne les étapes de notre travail, nous avons, en premier lieu, effectué
une phase d’étude théorique pour déterminer les problématiques liés à la réalisation
du projet pour ainsi fixer les solutions que nous devons adapter. En second lieu, nous
avons définit les services fournis par le Group4Success en détaillants les solutions à
faire pour satisfaire les quatres objectifs de notre application. En troisième lieu, nous
avons modélisé l’application par rapport aux besoins fonctionnels et techniques de notre
scénario de base. Après quoi, en quatrième partie, nous sommes passés à la réalisation de
l’application en faisant soin de choisir l'outil idéal pour se faire. Dans la dernière étape, nous avons présentés l'application à travers des imprimés écrans et enfin le déploiement de celle-ci dans un serveur distant.
\medskip

L'application ainsi développée fournit des services pour répondre aux besoins évolutifs des établissements d'enseignement supérieur. Elle permet aux établissements abonnés, possédant des clés d'accès à l'application, de gérer, d'une part, les gestions des employés et des étudiants. Ces gestions peuvent être, la création des fiches employeur, gestion des emplois, gestion de rémunération, traitement de masse, contrôles des personnelles et exportation des données, la gestion des formations, évaluation du parcours.
D'autre part, l'outil gère l'établissement et l'administration, à savoir : la gestion des inscriptions, suivis de l'institution, gestion des locaux, automatisation des procédures courantes, outil de tableau de bord.
\medskip

Malgré quelques difficultés à mettre en œuvre tous les modules, ce travail nous a présenté une
réelle occasion d'apprendre et de faire un premier pas vers le monde du SaaS, toujours en
évolution. Ce travail fut très instructif, car l'utilisation d'Angular change la manière de navigation des utilisateurs. C'est plus fluide, une meilleure gestion de contenu dynamique et diminution considérable de la vitesse de chargement des pages. En effet, le nombre d'accès au serveur est fortement diminué car la communication se fait majoritairement en mode asynchrone.
\medskip

À la fin de cet ouvrage, il nous est important de remarquer que des éventuelles améliorations doivent être apportés au présent projet. Dans la gestion des étudiants: nous pouvons envisager d'implanter un outil pour simulation pour auto-évaluer les progrès des étudiants. Pour l'exportation des fichiers, nous pouvons ajouter d'autres type de format autre que le PDF. Toutefois, nous estimons que les travaux demandés dans le cahier de charge ont été accompli à l'exception de la mise en œuvre de quelques modules qui suscitent d'autres études et analyses.