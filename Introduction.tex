%%%%%%%%%%%%%%%%%%%%%%%%%%%%%%%%%%%%%%%%%%%%%%%%%%%%%%%%%%%%%%%%%%%%%%%%%%%%%%%%%%%%%%%%%%%%%
%%									   Introduction		    							   %%
%%%%%%%%%%%%%%%%%%%%%%%%%%%%%%%%%%%%%%%%%%%%%%%%%%%%%%%%%%%%%%%%%%%%%%%%%%%%%%%%%%%%%%%%%%%%%
\chapter*{Introduction générale}
\addstarredchapter{Introduction générale}
De nouvelles dynamiques modifient rapidement l'économie de l'enseignement supérieur. L'élargissement des choix éducatifs, la globalisation et l'apparition de modèles agiles resserrent la concurrence des offres pour les
étudiants tandis que les organismes d'accréditations publiques de leur côté,
tiennent les institutions garantes de leur réussite. La pression de la scolarité,
la réduction des budgets et les attentes plus élevées des différents acteurs
poussent l'ensemble de l'écosystème à trouver une réponse adaptée à ce
nouveau cadre économique.
\medskip

Etudiants, corps professoral et personnel administratif attendent davantage de la révolution
technologique dont ils font par ailleurs l'expérience dans leur vie pour remplacer les outils 
dépassés encore utilisés dans leur cadre scolaire et universitaire.
\medskip

Ceci est encore plus vrai pour les systèmes d'information étudiants (Student Information Systems - SIS) qui remonte à une période bien antérieure à l'émergence des nouveaux modèles économiques de l'enseignement et des technologies disruptives d'aujourd'hui. Ces systèmes hérités perdurent uniquement grâce à des adaptations permanentes, qui rendent leur utilisation complexe et leur maintenance coûteuse.
\medskip

Le Group4Success va bouleverser cet état de fait. Conçu pour répondre aux besoins évolutifs des institutions modernes, Group4Success simplifie et réduit vos coûts de support informatique, apporte à la fois à vos étudiants une utilisation mobile intuitive et vous soutient dans les initiatives stratégiques grâces  toutes les données exploitables avec des possibilités d'analyses. Chaque acteur peut intervenir et agir à son niveau tout de suite et là où il se trouve. En d'autres termes, Group4Success permet d'augmenter notablement le retour sur investissement du remplacement de votre ancien système d'information. En bref, le Group4Success représente une alternative qui a tout pour changer la donne.
\medskip

Ce présent travail de projet consiste à développer ledit "Group4Success" qui est une application \glsentryshort{saas} (Software as a Service ou logiciel en tant que service) destiné pour alléger les efforts de gestion de personnel, soutenir la croissance de l'établissement, stimuler la réussite des étudiants et améliorer l'efficacité administrative. Comme il s'agit d'un SaaS, l'application et infrastructure sous-jacente sont intégralement hébergées à distance et administrées par les soins de chaque établissement.
\medskip

Par rapport a une application informatique classique, le SaaS est installés sur des serveurs distants plutôt que sur la machine de l'utilisateur. Les clients ne paient pas de licence d'utilisation pour une version, mais utilisent librement le service en ligne ou le plus généralement payent un abonnement. L'application demande une très grande protection des données, une grande rapidité et fluidité d'utilisation supportant un très grand nombre d'utilisateur ainsi qu'une très grande flexibilité avec possibilités d'extension (ajouts de modules complémentaires).
\medskip

Pour s'approprier au mieux ce projet, il est essentiel, dans un premier temps de faire une étude approfondie afin de cerner les besoin technique de l'application. En commençant par l'étude théoriques sur les systèmes et services des établissements d'enseignement supérieur et l'étude sur le fonctionnement d'un Saas.
\medskip

La deuxième partie va se consacrer sur la mise en place et définition des services fournis par le Group4Success. Cette partie sera subdivisé en quatre dont premièrement pour "alléger les efforts de gestion de personnel", deuxièmement pour "soutenir la croissance de l'établissement", troisièmement pour "stimuler la réussite des étudiants" et enfin quatrièmement pour "améliorer l'efficacité administrative".
\medskip

Dans la troisième partie, nous allons entamer à la phase d'analyse des besoins et du domaine du projet ainsi que la modélisation dynamique et stitique du système. Issue des études théorique, cette partie va décrire de manière visuelle et graphique les besoins et, les solutions fonctionnelles et techniques de notre projet.
\medskip

Puis dans la quatrième partie nous verrons les détails de la programmation, la conception du logiciel ainsi que les choix des langages utilisés. Enfin la dernière partie se concentre uniquement aux présentation, teste et déploiement de l'application ainsi développée. Puis définir les mise en œuvre des services fournis selon la politique d'un SaaS.



