%%%%%%%%%%%%%%%%%%%%%%%%%%%%%%%%%%%%%%%%%%%%%%%%%%%%%%%%%%%%%%%%%%%%%%%%%%%%%%%%%%%%%%%%%%%%%
%%									   Chapitre 1	    								   %%
%%%%%%%%%%%%%%%%%%%%%%%%%%%%%%%%%%%%%%%%%%%%%%%%%%%%%%%%%%%%%%%%%%%%%%%%%%%%%%%%%%%%%%%%%%%%%
\chapter{PRESENTATION DE LA POLYCLINIQUE
  UNIVERSITAIRE NEXT}
\minitoc
\newpage
%%%%%%%%%%%%%%%%%%%%%%%%%%%%%%%%%%%%%
	\section{Historique}
  	
  	La NEXT onlus a été fondée en 1998 par un chercheur scientifique, Luigi BELLINI, qui, à la
  	suite d'un voyage à Madagascar, profondément touché par les terribles conditions de vie de
  	la majorité (80\%) des personnes vivant avec moins d'un dollar par jour, a décidé
  	d'entreprendre une action pour aider ces gens, en consacrant toutes ses énergies et
  	ressources à ce but.
  	\medskip
  	
  	Le travail d'organisation et d'assistance s'est progressivement développé avec une
  	particulière attention dans le domaine de la santé dans le Nord de Madagascar, où ont été
  	accomplies des oeuvres importantes qui interagissent avec le système actuel de la Santé
  	Publique
  	\medskip
  	
  	L'activité de la NEXT, développée au début avec ses propres moyens financiers et sans le
  	soutien d'organismes extérieurs, a été reconnue par l'État Italien, qui, en Octobre 2006, lui a
  	accordé le statut juridique d'ONG (Organisation Non Gouvernementale).
  	\medskip
  	
  	Le professeur Luigi BELLINI a progressivement abandonné ses activités en Italie, ses
  	travaux de recherche à l'Université et professionnel, même sa collaboration scientifique avec
  	l'Agence spatiale européenne. Il a décidé de réaliser un centre de diagnostic médical à
  	Madagascar, dénommé « Le Samaritain ». Le centre, d'une superficie de 3 000 m2,
  	comprenant des laboratoires de radiologie et d'échographie, a débuté ses activités, en 2006.
  	Il est doté d'équipements modernes de niveau européen.
  	\medskip
  	
  	En constatant que le centre de diagnostic à lui seul ne suffit pas, l'ONG a mis sur les rails la
  	première clinique de maternité et de chirurgie dans le Nord. Une grande réalisation, unique à
  	Madagascar. Les travaux de construction ont débuté en 2009 : ce fut les premières
  	fondations de la Polyclinique universitaire NEXT.
  	
  	\medskip
  	
  
		
		\section{Localisation et  information sur le complex hospitalier}
		
		
		\subsection{localisation}
		Le Centre "Le Samaritain" ainsi que la Clinique de Maternité et Chirurgie se trouvent dans la
		région DIANA, située au Nord de Madagascar.
		(Pour arriver au complexe sanitaire il faut s'engager sur la route nationale 6 (RN6) et
		procéder 200 m avant de tourner à gauche vers « Rue de la Fraternité ».)
		
		\subsection{ Information concernant le complexe hospirtalier}
		
		L'objectif de la NEXT était d'assurer l'assistance médicale aux gens de la Région
		d'Antsiranana et de toutes les Provinces du Nord de Madagascar par la construction et
		l'équipement d'une Clinique, située à côté du Centre de Diagnostic Médical "Le Samaritain".
		Cette nouvelle structure à trois étages abritera les départements de maternité, chirurgie
		générale et médecine et sera équipée de :
		
		\begin{itemize}
			\item une salle opératoire principale

			\item une salle opératoire d'urgence ou secondaire

			\item une salle d'accouchement
			
			\item une salle de stérilisation

			\item une salle postopératoire
			
			\item un centre d'hémodialyse
			
			\item  un service d'urgences.
		\end{itemize}
		
		
		\section{Plan hospitalier}
		L'établissement de santé est reparti comme suit:
	
		\medskip
		\begin{tabular}{|c|c|c|}
		\hline  REZ-DE-CHAUSSÉE & 1ER ETAGE : & 2EME ETAGE \\ 
		\hline Triage-urgence & Bloc opératoire & Médecine \\ 
		 Hemodialyse & Chirurgie  & Pharmacie \\ 
		 Direction-administration & Gynéco-obstétrique & Salle de classe \\ 
		  & Maternité & Présidence \\ 
		  & Salle d'accouchement &  \\ 
		\hline 
		\end{tabular} 
		
