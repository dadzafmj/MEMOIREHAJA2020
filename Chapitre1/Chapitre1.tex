%%%%%%%%%%%%%%%%%%%%%%%%%%%%%%%%%%%%%%%%%%%%%%%%%%%%%%%%%%%%%%%%%%%%%%%%%%%%%%%%%%%%%%%%%%%%%
%%									   Chapitre 1	    								   %%
%%%%%%%%%%%%%%%%%%%%%%%%%%%%%%%%%%%%%%%%%%%%%%%%%%%%%%%%%%%%%%%%%%%%%%%%%%%%%%%%%%%%%%%%%%%%%
\chapter{Études théoriques}
\minitoc
\newpage
%%%%%%%%%%%%%%%%%%%%%%%%%%%%%%%%%%%%%
\section{Les établissements d'enseignement supérieur}
\subsection{Introduction}
L'enseignement supérieur désigne généralement l'éducation dispensée par les universités \cite{etudeSup}, avec un système plus dual de grandes écoles et d'autres institutions décernant des grades universitaires ou autres diplômes de l'enseignement supérieur.
\medskip

Ces études visent à acquérir un niveau "supérieur" de compétences, généralement via une inscription ou concours d'entrée, un cursus ponctués par des examens. 
\medskip

Ces études se déroulent souvent autour de campus, dans un système public ou privé selon les cas, et souvent catégorisés en « sciences dures », « sciences de l'ingénieur » et « sciences humaines et sociales ».
\medskip

Les types de diplômes évoluent avec le temps, mais tendent à s'homogénéiser aux niveaux européens et internationaux avec les licences, des masters, et des doctorats pour faciliter les systèmes d'équivalence ou de reconnaissance mutuelle de diplômes.
\medskip

Cette formation inclut des cours et des stages, et peut également comporter des participations à la recherche scientifique (notamment au niveau du doctorat) et intégrer de la formation continue, mais accepte souvent aussi des « auditeurs libres » qui viennent simplement accroître leur culture générale ou spécialisée.

\subsection{Objectifs}
Les établissement de l'enseignement supérieur contribue : 
\medskip

\begin{itemize}
	\item au développement de la recherche, support nécessaire des formations dispensées, et à l'élévation du niveau scientifique, culturel et professionnel de la nation et des individus qui la composent ;
	\item à la croissance régionale et nationale dans le cadre de la planification, à l'essor économique et à la réalisation d'une politique de l'emploi prenant en compte les besoins actuels et leur évolution prévisible ;
	\item à la réduction des inégalités sociales ou culturelles et à la réalisation de l'égalité entre les hommes et les femmes en assurant à toutes celles et à tous ceux qui en ont la volonté et la capacité l'accès aux formes les plus élevées de la culture et de la recherche ;
	\item à la construction de l'espace européen de la recherche et de l'enseignement supérieur.
\end{itemize}
\medskip

Les missions du service public de l'enseignement supérieur sont :
\medskip

\begin{itemize}
	\item La formation initiale et continue ;
	\item La recherche scientifique et technologique, la diffusion et la valorisation de ses résultats ;
	\item L'orientation et l'insertion professionnelle ;
	\item La diffusion de la culture et l'information scientifique et technique ;
	\item La participation à la construction de l'Espace européen de l'enseignement supérieur et de la recherche ;
	\item La coopération internationale.
\end{itemize}

\subsection{La formation initiale et continue}
Le service public de l'enseignement supérieur :
\medskip

\begin{itemize}
	\item [\textbullet] offre des formations à la fois scientifiques, culturelles et professionnelles,
	\item [\textbullet] accueille les étudiants et concourt à leur réussite et à leur orientation,
	\item [\textbullet] dispense la formation initiale,
	\item [\textbullet] participe à la formation tout au long de la vie,
	\item [\textbullet] assure la formation des formateurs.
\end{itemize}
\medskip

L'orientation des étudiants comporte une information sur le déroulement des études, sur les débouchés, sur les passages possibles d'une formation à une autre.
\medskip

La formation continue s'adresse à toutes les personnes engagées ou non dans la vie active. Organisée pour répondre à des besoins individuels ou collectifs, elle inclut l'ouverture aux adultes des cycles d'études de formation initiale, ainsi que l'organisation de formations professionnelles ou à caractère
culturel particulières.
\medskip

Le service public de l'enseignement supérieur met à disposition de ses usagers des services et des ressources pédagogiques numériques.

\section{Principes du LMD}
Le LMD (\glsentrylong{lmd}) désigne un ensemble de mesures modifiant le système d' n enseignement supérieur pour l'adapter aux standards européens. Elle met en place principalement une architecture basée sur trois grades : licence, master et doctorat ; une organisation des enseignements en semestres et unités d'enseignement \cite{reformeLMD}.
Le système LMD devient donc un système international d'harmonisation des cursus et des
diplômes, favorisant la mobilité internationale.
\medskip

Le passage au  système se fait progressivement et nécessite une démarche concertée entre les différents acteurs des universités (enseignants-chercheurs, étudiants, structures pédagogiques des universités et des établissements et les représentants de l'environnement socio-économique et culturel).  A Madagascar, il est tant convoité par les établissements supérieurs, mais il reste un parcours encore difficile à maîtriser.
\medskip

En effet, le système présente plusieurs avantages surtout pour les étudiants, à ne citer qu'il offre à celui-ci la possibilité de restructurer son parcours en cours de formation tout en ayant la possibilité de s'insérer dans le monde professionnel avec n'importe quel niveau de formation. Bien sur, les diplômes sont reconnus à l'international.

\subsection{Terminologie}
Le système LMD est fondée sur trois grades de références, mais désigne aussi trois cycles de formation et trois diplômes nationaux : 

\begin{itemize}
	\item[\textbullet] L = Licence (Bac + 3) $ \Longrightarrow $ L1, L2, L3 $ \Longrightarrow $ 6 semestres : S1, S2, S3, S4, S5, S6
	\item[\textbullet] M = Master (Bac + 5) $ \Longrightarrow $ M1, M2 $ \Longrightarrow $ 4 semestres : S7, S8, S9, S10
	\item[\textbullet] D = Doctorat (Bac + 8)
\end{itemize}

\begin{itemize}
	\item La Licence et le Master peuvent être à vocation Académique, Recherche ou Professionnelle. 
	\item Dans le système LMD, un grade universitaire (Licence, Master, Doctorat) sanctionne la fin d'un cycle.
	\item Au sein des cycles Licence et Master, les formations sont organisées en semestre.
\end{itemize}

\subsection{Domaines de formation}
Les domaines constituent le cadre général de l'offre de formation de l'établissement. Ils doivent ainsi représenter des ensembles cohérent fédérant les grands champs de compétences pédagogiques et scientifiques de l'établissement \cite{LMD}.

Madagascar à 6 domaines de formation dont : 
\begin{itemize}
	\item SCIENCES ET TECHNOLOGIE pour la faculté des Sciences
	\item ARTS, LETTRES et SCIENCES HUMAINES (ALSH) pour la faculté des Lettres et Sciences Humaines (FLSH)
	\item SCIENCES DE LA SOCIÉTÉ pour la faculté de Droit, d'Economie, de Gestion et de Sociologie (DEGS)
	\item SCIENCES DE LA SANTÉ pour la faculté de Médecine
	\item SCIENCES DE L'INGENIEUR pour l'Ecole Supérieure Polytechnique (ESP)
	\item SCIENCES DE L'ÉDUCATION pour l'Ecole Normale Supérieure (ENS) et l'Ecole Normale Supérieure pour l'Enseignement Technique (ENSET)
\end{itemize}

\subsection{Les mentions}
Le domaine de formation est structuré en plusieurs Mentions.
La Mention couvre un champ scientifique relativement large qui permet d'identifier le thème majeur de la formation. Elle permet de faire apparaitre la finalité soit académique, recherche soit professionnelle.

\subsubsection{Exemple 1}
\underline{Domaine :} SCIENCES DE L'INGÉNIEUR \\
\underline{Mentions} (grades Licence et Master) : 
\begin{itemize}
	\item Génie Mécanique
	\item Hydraulique et Énergétique
	\item Génie Civil
	\item Science et Technologie de l'Information et de la Communication
	\item Génie Électrique
\end{itemize} 

\subsubsection{Exemeple 2}
\underline{Domaine :} SCIENCES ET TECHNOLOGIE \\
\underline{Mentions} (grades Licence et Master) :
\begin{itemize}
	\item Chimie
	\item Physique et Application
	\item Biologie
	\item Geosysteme et Évolution
	\item Mathématique et Informatique
\end{itemize}

\subsection{Les parcours}
L'offre de formation est organisée sous forme de parcours diversifiés et adaptés. Un parcours de formation est un ensemble cohérent d'Unités d'Enseignements (UE) articulées entre elles de façon à offrir : 
\begin{itemize}
	\item D'une part, une progression pédagogique adaptée en fonction de l'origine de l'étudiant et de son projet personnel, 
	\item D'autre part, des possibilités de réorientation ou de complément de formation à chacun des paliers.
\end{itemize}

\subsubsection{Exemple 1}
\underline{Domaine :} SCIENCES DE L'INGÉNIEUR \\
\underline{Mentions} (grades Licence et Master) : SCIENCES ET TECHNOLOGIE DE L'INFORMATION ET DE LA COMMUNICATION

\begin{table}[h]
	
	\caption{Organisation des études mentions STIC, parcours}
	\label{Organisation des études mentions STIC, parcours}
	
	\begin{center}
		{\renewcommand{\arraystretch}{1.6} % horizontal padding
			\begin{tabular}{|c|>{\centering}m{4cm}|m{4cm}|}
				\hline \textbf{Semestres} & \multicolumn{2}{c|}{Parcours} \\ 
				\hline S10, S9, S8 & Électronique et Informatique Industrielle (STIC) & Télécommunication et Réseaux (TR) \\ 
				
				\hline S7 & \multicolumn{2}{c|}{Tronc commun STIC} \\
				\hline S6, S5, S4, S3 & \multicolumn{2}{c|}{STIC} \\
				\hline S2, S1 & \multicolumn{2}{c|}{Tronc Commun Polytechnique} \\
				\hline
			\end{tabular} 
		} \quad
	\end{center}
	
\end{table}

\clearpage

\subsubsection{Exemple 2}
\underline{Domaine :} SCIENCES ET TECHNOLOGIE \\
\underline{Mentions} (grades Licence) : BIOLOGIE

\begin{table}[h]
	
	\caption{Organisation des études mentions Biologie, parcours}
	\label{Organisation des études mentions Biologie, parcours}
	
	\begin{center}
		\hspace*{-0.5cm}
		{\renewcommand{\arraystretch}{1.6} % horizontal padding
			\begin{tabular}{|c|>{\centering}m{2.2cm}|>{\centering}m{2cm}|>{\centering}m{2.2cm}|>{\centering}m{2.6cm}|m{2.5cm}|}
				\hline \textbf{Semestres} & \multicolumn{5}{c|}{Parcours} \\ 
				\hline S6, S5 & Biologie des Organismes et des Ecosystèmes (BOE) & Biochimie et Biologie Moléculaire (BBM) 
				& Entomologie Appliquée (ENTAP) & Physiologie, Pharmacologie, Cosmétologie (PPC) & Anthropologie Biologique et Evolution (ABE)\\ 
				
				\hline S4, S3 & \multicolumn{5}{c|}{Tronc commun Biologie(TCB)} \\
				
				\hline S2, S1 & \multicolumn{5}{c|}{Portail Sciences de la vie et de la terre (PSVT)} \\
				\hline
			\end{tabular} 
		} \quad
	\end{center}
	
\end{table}

\subsection{Les crédits}
Un crédit est une unité de mesure, une unité de compte, exprimant la valeur donnée à une Unité d'Enseignement ou à un Elément Constitutif (\glsentryshort{ec}) d'une UE.
Le système de crédit est applicable à toutes les activités d'enseignement y compris stages, mémoires, projets, travail personnel et aussi à toutes les formes d'enseignement (présentiel, ouvert, à distance, en ligne,...).
\medskip

Au sein des cycles Licence et Master, les formations sont organisées en semestres et qu'un crédit équivaut à 10 heures de travail (travail pésentiel, travail personnel) :
\begin{itemize}
	\item Offres de formation dans le grade LICENCE : 
	\begin{itemize}
		\item[\textbullet] Duréé d'études : 06 semestres
		\item[\textbullet] Total des crédits : 180 dont 30 crédits par semestre
	\end{itemize}
	\item Offres de formation dans le grade MASTER : 
	\begin{itemize}
		\item[\textbullet] Duréé d'études : 04 semestres
		\item[\textbullet] Total des crédits : 120 dont 30 crédits par semestre
	\end{itemize}
\end{itemize}

\subsubsection{Exemple} Pour l'Ecole Supérieure Polytechnique : 
\begin{itemize}
	\item Domaine : SCIENCES DE L'INGÉNIEUR
	\item Grade : MASTER
	\item Mention : GÉNIE ÉLECTRIQUE
	\item Parcours en M2 de S8 : Production, Transport et Distribution d'Energie
	\item Crédit : 30 (répartis dans les types d'intervention « Cours, TD, TP, Stage, etc. » des UE)
\end{itemize}

\subsection{L'Unité d'Enseignement (UE)}
L'Unité d'Enseignement (UE) est la base du dispositif LMD. En effet, toutes les études sont organisées en unités d'enseignement.\\
Une UE peut être une matière ou un ensemble de matières choisies pour leur cohérence dans cet ensemble.

\subsubsection{Différentes catégories d'UE}
\begin{itemize}
	\item UE fondamentales qui sont obligatoires et doivent être suivies par tous les étudiants
	\item UE complémentaires qui complètent les UE fondamentales du parcours choisi par l'étudiant et doivent être prises obligatoirement afin de valider le parcours de l'étudiant.
	\item UE libre qui sont au choix.
\end{itemize}
\medskip

Les UE peuvent être donc : 
\begin{enumerate}
	\item Obligatoires ou optionnelles (facultatives)
	\item Transférables d'un parcours à l'autre
	\item Capitalisables puisque toute validation d'UE est acquise quelle que soit la durée d'un parcours.
\end{enumerate}

\subsubsection{Exemple} En Faculté des Sciences : 
\begin{itemize}
	\item Domaine : SCIENCES ET TECHNOLOGIE
	\item Grade : LICENCE
	\item Mention : CHIMIE
	\item Parcours en L1 de S2 : Formation de base en chimie (Tronc Commun)
	\item Nombre de UE dans ce parcours L1/S2 : 06
	\begin{itemize}
		\item [\textbullet] Chimie organique physique $ \Longrightarrow $ 5 crédits
		\item [\textbullet] Initiation à la physico-chimie des solutions aqueuses $ \Longrightarrow $ 5 crédits
		\item [\textbullet] Chimie inorganique $ \Longrightarrow $ 5 crédits
		\item [\textbullet] Initiation à la géométrie et éléments d'analyse $ \Longrightarrow $ 4 crédits
		\item [\textbullet] Physique 1 $ \Longrightarrow $ 4 crédits
		\item [\textbullet] Initiation à la Science de la Vie et de la Terre $ \Longrightarrow $ 7 crédits
	\end{itemize}
\end{itemize}
Pour chaque UE les crédits sont répartis dans des types d'intervention « Cours, TD, TP, Stage, etc. » et le total de crédits pour cette UE est égal à 30.

\subsection{L'Elément Constitutifs (EC)}
Chaque contenu d'une UE est appelé Elément Constitutif (EC).\\
Chaque EC de l'UE appelé aussi matière ou composante de l'UE est évalué selon un examen semestriel.

\subsection{Les avantages du systeme LMD}
Le système LMD comporte une série d'avantages incontestables qui ont fondé son acceptation et son adoption à l'échelle mondiale. Leur perception varie selon le contexte, les réalités dans les pays qui ont déjà adopté le LMD ou qui sont en voie de le faire. 

\subsubsection{Exemple}
\begin{enumerate}
	\item Pour l'apprenant :
	\begin{itemize}
		\item [\textbullet] Réduction de la durée d'études liée à la semestrialisation
		\item [\textbullet] Possibilité de poursuivre des études dans une autre université ou dans un pays appliquant le LMD
		\item [\textbullet] Possibilité d'interrompre sa formation et de la reprendre par la suite en conservant le bénéfice des crédits qu'il a validés auparavant
		\item [\textbullet] Professionnalisation facilitée par les UE, stages, projets
		\item [\textbullet] Diversité des parcours proposés grâce à un système de passerelles
		\item [\textbullet] Choix de son parcours de formation en fonction de ses projets d'études et professionnels, etc...
	\end{itemize}
	\item Pour l'institution : 
	\begin{itemize}
		\item [\textbullet] Une offre de formation moderne, attractive et compétitive
		\item [\textbullet] La possibilité de modifier plus facilement l'offre de formation en fonction des besoins et des valorisations de la demande (ajouter, changer le statut d'un module,...)
		\item [\textbullet] Un encadrement rapproché qui implique des effectifs réduits et un contrôle continu des connaissances.
		\item [\textbullet] La professionnalisation des enseignements oblige universités à être d'avantage attentives dans leur fonctionnement pédagogique aux exigences du marché du travail (profils techniques pour les étudiants) et pour les emplois à pourvoir, etc...
	\end{itemize}
	\item Pour la société : 
	\begin{itemize}
		\item [\textbullet] Ouverture de l'Université sur la société
		\item [\textbullet] Augmentation du nombre de métiers
		\item [\textbullet] Adaptation des formations aux besoins sociaux
		\item [\textbullet] Meilleure participation de l'Université au développement de la société
	\end{itemize}
\end{enumerate}

\section{Gestion des ressources humaines}
La gestion des ressources humaines est plus que jamais un pilier de la performance de l'établissement supérieur. Les questions de gestion de compétences, de recrutement, de rémunération ou même d'implication de ses collaborateurs se positionnent au centre des préoccupations des employeurs. Pour définir simplement la notion des ressources humaines, on peut dire qu'il s'agit d'un service piloté par un Directeur des ressources humaines qui est en charge de plusieurs services. Il s'occupe par exemple de la gestion du personnel, des administrations, de la communication.
\medskip

\subsection{Définition}
La \gls{grh} est un sous système du management d'une entreprise; elle est la facette humaine qui permet d'organiser le travail et de traiter les travailleurs de manière qu'ils puissent faire valoir autant que possible leur capacité à fin d'obtenir un rendement maximal pour eux-mêmes et pour leur groupe \cite{gestionRH}.
\medskip

C'est l'ensemble de mesures et d'activités impliquant des ressources humaines et ayant pour objectif d'améliorer l'efficacité et la performance des individus et de l'organisation. C'est aussi un ensemble de pratiques ayant pour objectif de mobiliser et de développer les ressources humaines pour une plus grande efficacité et efficience, en soutien de la stratégie d'une organisation (association, entreprise, administration).

La GRH est une dimension de la gestion de l'entreprise, elle existe le plus souvent en tant que fonction à part entière.
Même si des tendances récentes la révèlent sous un angle de plus en plus dispersé et partagé, elle a un certain nombre de mission à accomplir et une série d'activités à organiser et à gérer, c'est ce que nous allons voir dans les sections suivantes \cite{gestionRHBook}. 

\subsection{Gestion des carrières}
La gestion des carrières est l'ensemble du cheminement professionnel de l'individu qui va s'étendre durant la totalité de sa vie active au seins d'une organisation, elle permet de gérer de nombreux domaines : 

\begin{itemize}
	\item[\checkmark] Le recrutement.
	\item[\checkmark] La sélection.
	\item[\checkmark] La formation.
	\item[\checkmark] La motivation et l'implication du personnel.
	\item[\checkmark] La gestion de la paie et des rémunérations.
	\item[\checkmark] Les relations sociales.
	\item[\checkmark] Les conditions de travail.
	\item[\checkmark] L'évaluation des performances et compétences.
\end{itemize}

\subsection{Recrutement ou gestion des postes}
Une vision globale des postes : \\
Les caractéristiques du poste à pourvoir sont gérées : description, responsable, lieu, mission, début d'activité, type de contrat, etc...

\begin{itemize}
	\item[\checkmark] Le recrutement.
	\item[\checkmark] Recherche multicritères sur les postes à pouvoir paramétrable.
\end{itemize}

\subsection{Gestion des candidats et leur sélection}
Vivier de candidat : dossier personne, CV, disponibilité, rémunération souhaitée, origine. Le contenu du dossier individuel est paramétrable avec le générateur de formulaires intégré.
\medskip

Impression des listes et fiches des candidats : elle est entièrement personnalisable grâce au générateur de rapports intégré.
\medskip

Documents : en plus du CV saisi, il est possible d'attacher au candidat des fichiers (Word, Excel, PDF, etc...) pour stocker le CV, diplômes, certificats, etc.

\subsection{Gestion de la correspondance}
Automatisation de la correspondance : envoi d'email ou impression de courrier Word, paramétrable à partir de modèles et des statuts des candidatures.
\medskip

Archivage de la correspondance dans le dossier du candidat.
\medskip

Automatisation du processus administratif : Mise à jour automatique des status du candidat selon l'évolution de son dossier.

\subsection{Gestion des supports}
Gérer les supports tout naturellement

\begin{itemize}
	\item[\checkmark] Gestion des catalogues regroupant les cours destinés à une catégorie de personnes et pour une période donnée.
	\item[\checkmark] Gestion des tableaux de bord de l'activité formation pour le management et pour les RH
	\item[\checkmark] Gestion des saisies du budget de formation par service et suivi du solde, détaillé ou consolidé.
\end{itemize}

\subsection{La gestion de la paie et des rémunérations}
La gestion de la paie constitue un aspect important de l'administration des salariés. Matérialisant la relation entre l'employeur et le salarié. Le salaire, contrepartie de la prestation du travail effectuée par le salarié, constitue un coût pour l'établissement.
\medskip

La fixation de la rémunération prend en compte la complexité des tâches à effectuer, mais également les conditions de travail.
\clearpage

\section{Le logiciel en tant que service (SaaS)}
Le SaaS (Software as a Service) est un concept assez récent qui permet aux entreprise de s'abonner à un logiciel à distance au lieu de les acquérir et de devoir les installer sur leur propre matériel informatique.

\subsection{Définition}
Le logiciel en tant que service ou \gls{saas} \cite{SaaS} est un modèle d'exploitation commerciale des logiciels dans lequel ceux-ci sont installés sur des serveurs distants plutôt que sur la machine de l'utilisateur. Les clients ne paient pas de licence d'utilisation pour une version, mais utilisent librement le service en ligne ou le plus généralement payent un abonnement.
\medskip

Le logiciel en tant que service (\gls{saas}) est donc la livraison conjointe de moyens, de services et d'expertise qui permettent aux entreprises d'externaliser intégralement un aspect de leur système d'information (messagerie, sécurité,...) et de l'assimiler à un coût de fonctionnement plutôt qu'à un investissement. Le contrat de services est essentiel pour définir le niveau de qualité de service (\glsentryshort{sla}). Le logiciel en tant que service (\gls{saas}) peut être vu comme l'équivalent commercial de l'architecture orientée service (\glsentryshort{soa}).
\medskip

Les solutions logicielles en tant que service (SaaS) sont principalement développées à destination d'entreprises. Depuis quelques années le marché des SaaS est en très forte croissance. Ces logiciels présentent pour les entreprises divers avantages et inconvénients.

\subsection{Fonctionnement}
Dans ce type de service, des applications sont mises à la disposition des consommateurs. Les applications peuvent être manipulées à l'aide d'un navigateur web ou installées de façon locative sur un PC, et le consommateur n'a pas à se soucier d'effectuer des mises à jour, d'ajouter des patches de sécurité et d'assurer la disponibilité du service
\medskip

Le SaaS permet à une entreprise de ne plus installer d'applications sur ses propres serveurs mais de s'abonner à des logiciels en ligne et de payer un prix qui variera en fonction de leurs utilisateurs effectives \cite{modeSaaS}.
\medskip

En utilisant le SaaS, l'entreprise n'hébergera pas ses applications et ne stockera pas ses données en interne.
\medskip

Il n'est donc pas nécessaire d'acquérir directement ces applications et posséder des serveurs pour les héberger. De plus, la maintenance et les mises à jour des applications seront gérées en externe par le prestataire.
\medskip

Les utilisateurs de l'entreprise devront simplement disposer d'un ordinateur et des codes d'accès au service en ligne pour pouvoir travailler.

\subsection{Les avantages}
La gestion en mode SaaS permet à une entreprise de bénéficier de nombreux avantages :
\medskip

\begin{itemize}
	\item Aucun logiciel à installer sur les matériels informatique de l'entreprise,
	\item Pas de données stockées en interne,
	\item Mise à jour des applications automatiques,
	\item L'application peut être utilisée partout et n'importe quant : il suffit d'une simple connexion internet et d'un navigateur web.
\end{itemize}
\medskip

Les deux premiers avantages cités en recourant au mode SaaS permettant en plus à l'entreprise de réaliser des économies sur ces investissements en matériel informatique.
\medskip

L'utilisation de solutions logicielles en tant que service (SaaS) en entreprise permet aussi un meilleur contrôle des charges techniques. L'ensemble des solutions techniques étant délocalisées le coût devient fixe, généralement fonction du nombre de personnes utilisant la solution SaaS. Le prix par utilisateur englobe le coût des licences des logiciels, de la maintenance et de l'infrastructure. Il revient à l'entreprise utilisatrice de faire son choix entre utilisation en SaaS, d'une part, et acquisition des licences puis déploiement en interne, d'autre part.
\medskip

Le mode SaaS est un service, dont le coût constitue une charge immédiatement déductible du résultat de l'entreprise, alors qu'un investissement est déduit du résultat par le biais d'amortissements étalés sur plusieurs années.
\medskip

Un avantage manifeste pour les entreprises est la rapidité de déploiement lorsque le logiciel SaaS correspond exactement au besoin (et qu'il ne nécessite aucune adaptation). Les solutions SaaS étant déjà pré-existantes le temps de déploiement est extrêmement faible.
\medskip

Un autre avantage pourrait être de réduire la consommation électrique en permettant la mutualisation des ressources sur des serveurs partagés par plusieurs entreprises (architecture \gls{multiTenant}) ainsi que l'usage d'un ordinateur à faible consommation muni d'un simple navigateur Web sans autres licences associées.

\subsection{Les inconvénients}
Le mode SaaS présente toutefois quelques inconvénients. Lors de la mise en place de solutions SaaS, les données relatives à l'entreprise cliente sont, généralement, stockées sur les serveurs du prestataire fournissant la solution. Lorsqu'il s'agit de données sensibles ou confidentielles, l'entreprise est obligée de prendre des dispositions contractuelles avec le fournisseur. Ainsi l'entreprise est dépendante du prestataire qui lui fournit le service. Que se passe-t-il si ce dernier vous fait défaut ? Dépose le bilan ? ou si un litige survient ?...
\medskip

La délocalisation des serveurs de la solution SaaS permet également un accès nomade aux données de l'entreprise. Cet accès entraîne un souci de sécurité de l'information lors du départ de collaborateurs. Il est indispensable d'avoir mis en place des procédures permettant, lors d'un départ, de supprimer l'habilitation de l'ancien collaborateur à accéder aux données de l'entreprise.
\medskip

Les migrations informatiques peuvent être compliquées puisqu'il faut basculer les données de la plate-forme d'un fournisseur vers celle d'un autre, avec divers problèmes associés (compatibilité, apparence pour le client, etc.). Dans le cadre du SaaS, le client se trouve lié à son fournisseur.
