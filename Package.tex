%%%%%%%%%%%%%%%%%%%%%%%%%%%%%%%%%%%%%%%%
%           Liste des packages         %
%%%%%%%%%%%%%%%%%%%%%%%%%%%%%%%%%%%%%%%%
%% Faux texte
\usepackage{blindtext}

%%%%%%%%%%%%%%%%%%%%%%%%%%%%%%%%%%%%%%%%%%%%%%%%%%%%%%%%%%%%%%%%%%%%%
%% Réglage des fontes et typo    
\usepackage[utf8]{inputenc}

\usepackage[top=2.5cm, bottom=2cm, left=3cm, right=2.5cm, headheight=15pt]{geometry} 

\usepackage[frenchb]{babel}
\usepackage{lmodern} 					% Pour changer le pack de police.
\usepackage[upright]{fourier}			% Police tsara
\usepackage{pdfpages}					% Inclure pdf
\usepackage{pdflscape}					% Permet d'utiliser des pages au format paysage

\setlength{\parindent}{0pt} 			% supprime l'indentation
\sloppy									%	
\hyphenpenalty 10000					% empeche la coupure des mots

\usepackage{moreverb} % Code source avec tabulation
%%%%%%%%%%%%%%%%%%%%%%%%%%%%%%%%%%%%%%%%%%%%%%%%%%%%%%%%%%%%%%%%%%%%%
% ABSTRACT


%%%%%%%%%%%%%%%%%%%%%%%%%%%%%%%%%%%%%%%%%%%%%%%%%%%%%%%%%%%%%%%%%%%%%
% En-tête et pied de pages
\usepackage{fancyhdr}

\pagestyle{fancy}
\lhead{}
\chead{}
\rhead{Conception et  réalisation du logiciel de gestion de la Polyclinique universitaire NEXT}
\lfoot{}
\cfoot{\thepage}
\rfoot{} 	
	
%%%%%%%%%%%%%%%%%%%%%%%%%%%%%%%%%%%%%%%%%%%%%%%%%%%%%%%%%%%%%%%%%%%%%
% Images 
\usepackage{graphicx} 		% Images
\usepackage{wrapfig} 		% Images

%%%%%%%%%%%%%%%%%%%%%%%%%%%%%%%%%%%%%%%%%%%%%%%%%%%%%%%%%%%%%%%%%%%%%
% Tableaux
\usepackage{multirow}
\usepackage{booktabs}
\usepackage{colortbl}
\usepackage{tabularx}
\usepackage{multirow}
\usepackage{threeparttable}

\addto\captionsfrench{\def\tablename{{\textsc{Tableau}}}}	% Renome 'table' en 'tableau'

\usepackage{supertabular}

\definecolor{grisclair}{gray}{0.8} % Définition d'une couleur gris

%%%%%%%%%%%%%%%%%%%%%%%%%%%%%%%%%%%%%%%%%%%%%%%%%%%%%%%%%%%%%%%%%%%%%
% Math
\usepackage{amsmath} 
\usepackage{amssymb} 
\usepackage{mathrsfs}
\usepackage{amsthm}
	
%%%%%%%%%%%%%%%%%%%%%%%%%%%%%%%%%%%%%%%%%%%%%%%%%%%%%%%%%%%%%%%%%%%%%   
% Table des matières
\usepackage[francais]{minitoc}		% Mini table des matières, en français
	\setcounter{minitocdepth}{2}	% Mini-toc détaillée (sections/sous-sections)
	
% Mettre Chapitre 1. "nom du chapitre" dans le table of contents
\usepackage{titletoc}
\titlecontents*{chapter} % <section type>
	[0pt] % <left>
	{\addvspace{1em}} %
	{\bfseries\chaptername\ \thecontentslabel\quad}% <numbered-entry-format}
	{} % <numbereless-entry-format>
	{\bfseries\hfill\contentspage} % <filler-page-format>

%%%%%%%%%%%%%%%%%%%%%%%%%%%%%%%%%%%%%%%%%%%%%%%%%%%%%%%%%%%%%%%%%%%%%
% Glossaire
\usepackage{glossaries}
\newglossary[nlg]{notation}{not}{ntn}{Notation}
\makeglossaries

\usepackage{hyphenat}
%\setacronymstyle{long-short}

%%%%%%%%%%%%%%%%%%%%%%%%%%%%%%%%%%%%%%%%%%%%%%%%%%%%%%%%%%%%%%%%%%%%%
% Biblio                        
%\usepackage{natbib}

%%%%%%%%%%%%%%%%%%%%%%%%%%%%%%%%%%%%%%%%%%%%%%%%%%%%%%%%%%%%%%%%%%%%%
%% Navigation dans le document   
\usepackage[pdftex,pdfborder={0 0 0},
			colorlinks=true,
			linkcolor=black,
			citecolor=black,
			pagebackref=false,
			]{hyperref} %Créera automatiquement les liens internes au PDF
			
%%%%%%%%%%%%%%%%%%%%%%%%%%%%%%%%%%%%%%%%%%%%%%%%%%%%%%%%%%%%%%%%%%%%%
%% Compilation
\usepackage{silence}
%
%% Virer les erreur dues à minitoc
\WarningFilter{minitoc(hints)}{W0023}
\WarningFilter{minitoc(hints)}{W0024}
\WarningFilter{minitoc(hints)}{W0028}
\WarningFilter{minitoc(hints)}{W0030}			