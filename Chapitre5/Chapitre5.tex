%%%%%%%%%%%%%%%%%%%%%%%%%%%%%%%%%%%%%%%%%%%%%%%%%%%%%%%%%%%%%%%%%%%%%%%%%%%%%%%%%%%%%%%%%%%%%
%%									   Chapitre 5	    								 %%
%%%%%%%%%%%%%%%%%%%%%%%%%%%%%%%%%%%%%%%%%%%%%%%%%%%%%%%%%%%%%%%%%%%%%%%%%%%%%%%%%%%%%%%%%%%%%
\chapter{Présentation et déploiement de l'application}
\minitoc
\newpage
%%%%%%%%%%%%%%%%%%%%%%%%%%%%%%%%%%%%%

\section{Pré-requis}
Pour faire tourner l'application, l'utilisateur devrait munir d'un périphérique possédant un simple navigateur web, tels un ordinateur, une tablette ou smartphone. Puis d'être abonnés à l'application et avoir les codes d'accès. Sachant que l'application est hébergée à distance, une connexion internet est indispensable. Une connexion ADSL (Asymmetric Digital Subscriber Line) de 8Mo/s sera largement suffisante pour cela. 

\section{Présentation de l'application}
Nous allons faire une brève présentation de l'interface aussi bien que les fonctionnalités
de l'application. Notons au tout début que c’est une application SaaS qui a
pour nomination : « Group4Success ». Cette application sera utilisé par différentes
établissement en utilisant le même URL.

\subsection{Fenêtre d'authentification}
\begin{figure}[h]
	\centering
	\includegraphics[width=0.55\linewidth]{"Chapitre5/images/authentification"}
	\caption{Fenêtre d'authentification}
	\label{Fenêtre d'authentification}
\end{figure}

Avant tout, l'établissement client doit impérativement être inscrit à un abonnement sur le site web de G4SM (Group4SuccessManagement) afin de recevoir un identifiant et un mot de passe pour que l’administrateur client puisse s'authentifier dans cette application.
\clearpage

\subsection{Fenêtre d'inscription}
Le site web de G4SM qui devrait servir de présentation et inscription à un abonnement n'est pas encore déployer en ligne. Pour cela, on peut tester l'application par l'intermédiaire du lien "Enregistrer un nouvel abonnement" accessible depuis l'écran d'authentification. Ce dernier ouvre une fenêtre d'inscription pour pouvoir créer un compte de test.

\begin{figure}[h]
	\centering
	\includegraphics[width=0.55\linewidth]{"Chapitre5/images/inscription"}
	\caption{Fenêtre d'inscription}
	\label{Fenêtre d'inscription}
\end{figure}

La validation des données dans les champs du formulaire est gérée par la méthode de réactive fourni par Angular. Cette méthode utilise des fonctions en JavaScript pour écouter les données saisie de l'utilisateur en temps réel.
\clearpage

\subsection{Les email de validation et de réinitialisation de mot de passe}
A partir de la fenêtre d'authentification on peut récupérer notre mot de passe en cas d'oubli. Pour ce faire, il faut cliqué sur le lien "Mot de passe oublié" puis de saisir l'adresse email. Après cela, le système envoi un email à ce dernier pour réinitialiser le mot de passe.
\medskip

\begin{figure}[h]
	\centering
	\includegraphics[width=1\linewidth]{"Chapitre5/images/emailReiniatiasationMDP"}
	\caption{Email de réinitialisation de mot de passe}
	\label{Email de réinitialisation de mot de passe}
\end{figure}

Le système envoi aussi un email de validation après chaque inscription pour vérifier l'authenticité de l'adresse email.

\begin{figure}[h]
	\centering
	\includegraphics[width=1\linewidth]{"Chapitre5/images/emailValidation"}
	\caption{Email de validation de l'adresse email}
	\label{Email de validation de l'adresse email}
\end{figure}
\clearpage

\subsection{Le tableau de bord}

\begin{figure}[h]
	\centering
	\hspace*{-1.8cm}
	\includegraphics[width=1.2\linewidth]{"Chapitre5/images/dashboard"}
	\caption{Tableau de bord}
	\label{Tableau de bord}
\end{figure}

Après authentification, l’utilisateur sera redirigé directement au tableau de bord orienté
rôles qui affiche les chiffres clés de l’établissement et une vue d’ensemble simple et directe
des anomalies. Le tableau de bord permettant de zoomer jusqu’au détail et d’utiliser
toutes les informations de données de l’établissement. L’architecture du tableau de bord
est adaptable et configurable, c’est-à-dire que l’on peut ajouter, supprimer ou déplacer
un éléments pour le façonner à notre besoin.
\medskip

La paramètre de l'application, à droite n'est disponible que pour l'administrateur, elle permet de modifier les informations de l'application tels que son nom et le logo.

\subsection{Liste et ajout d’utilisateur}
Pour chaque nouveau utilisateur ajouter, un mail de vérification et réinitialisation de mot de passe sera envoyé à son adresse qu’il puisse s’authentifier à l’application.
La modification ou suppression d’un compte utilisateur est accessible en passant par la
liste. Cette liste est dynamique, elle offre plusieurs options d’affichage, trie par colonne ou recherche d’un élément spécifique.

\begin{figure}[h]
	\centering
	\hspace*{-1.8cm}
	\includegraphics[width=1.2\linewidth]{"Chapitre5/images/ajoutUtil"}
	\caption{Ajout d'utilisateur}
	\label{Ajout d'utilisateur}
\end{figure}

\begin{figure}[h]
	\centering
	\hspace*{-1.8cm}
	\includegraphics[width=1.2\linewidth]{"Chapitre5/images/listeUtil"}
	\caption{Liste d'utilisateur}
	\label{Liste d'utilisateur}
\end{figure}

\clearpage

\subsection{Page d'erreur 404}
Au cas où un utilisateur essai d'accéder à un url non valide qui n'existe pas, il sera redirigé directement vers une page d'erreur 404.

\begin{figure}[h]
	\centering
	\hspace*{-1.8cm}
	\includegraphics[width=1.2\linewidth]{"Chapitre5/images/error404"}
	\caption{Page d'erreur 404}
	\label{Page d'erreur 404}
\end{figure}

\section{Les services backend de Firebase}
\subsection{Authentification}
Firebase fournit des services de backend pour gérer la gestion des utilisateurs. Avec Firebase Authentication, les mots de passe des utilisateurs sont enregistrés de manière sécurisée dans le cloud, haché avec l'algorithme SCRYPT en 64bits (Voir figure 5.10). Ces mots de passe haché n'est même pas accéssible à l'administreur de Firebase, ils sont consérvés en sécurité.


\section{Déploiement de l'application sur Firebase}
Après avoir développé une application il faut pouvoir la déployer sur un serveur pour la rendre disponible par les utilisateurs via un internet. Pour cela nous allons utiliser Firebase hosting comment hébergeur.

\subsection{Préparer l'application pour la production}
A la racine de notre application Angular, on exécute la commande suivante : 

\begin{verbatim}
	ng build -prod
\end{verbatim}

Celle-ci a pour effet de créer un répertoire \emph{dist}. C'est là que se trouve l'application compilée, optimisée et prête à être déployée.

\subsection{Installer Firebase-CLI}
Firebase-CLI est nécessaire pour configurer notre application sur firebase. Pour cela, dans l'invite de commandes en tant qu'Admninistrateur, il faut entrez la commande suivante : 

\begin{verbatim}
	npm install -g firebase-tools
\end{verbatim}

Une fois l'installation terminée, il faut se connecter avec notre compte Google : 

\begin{verbatim}
	firebase login
\end{verbatim}

Cette commande associe notre PC à notre compte Firebase et nous permet alors d'accéder à nos projets.

\subsection{Configurer l'application Angular}
Nous devions initialiser Firebase Hosting avant de pouvoir déployer notre application Angular.
Pour ce faire, on exécute la commande suivante :

\begin{verbatim}
	firebase init
\end{verbatim}

\subsection{Déploiement de l'application}
Après quelques configuration avec \emph{firebase init}, on peut finalement déployer notre application avec la commande suivante : 

\begin{verbatim}
	firebase deploy
\end{verbatim}

Le déploiement effectué, notre application est maintenant accessible via l'URL : \\ \emph{"g4sm-mad.firebaseapp.com/"}