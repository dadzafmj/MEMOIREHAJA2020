%%%%%%%%%%%%%%%%%%%%%%%%%%%%%%%%%%%%%%%%%%%%%%%%%%%%%%%%%%%%%%%%%%%%%%%%%%%%%%%%%%%%%%%%%%%%%
%%									   Chapitre 3	    								 %%
%%%%%%%%%%%%%%%%%%%%%%%%%%%%%%%%%%%%%%%%%%%%%%%%%%%%%%%%%%%%%%%%%%%%%%%%%%%%%%%%%%%%%%%%%%%%%
 \chapter{Programmation et réalisation du logiciel}
\minitoc
\newpage
%%%%%%%%%%%%%%%%%%%%%%%%%%%%%%%%%%%%%
\section{Introduction}
	  La réalisation et le codage de l'application est une étape cruciale et inévitable,  dans le développement de l'application, l'application ne se construit tout seul après la modélisation, pour que l'application ne soit pas devenir une simple rêve il faut le produire.
	  \medskip
	  Nous avons utilisé laravel pour le codage.
	  
	  
	  \section{Le framwork Laravel}
	  
	  Les applications Laravel sont installées et gérées avec Composer , un gestionnaire de
	  dépendance PHP. Il existe deux manières de créer une nouvelle application Laravel.
	  
	 Sur le terminal:
	  composer create-project laravel/laravel [foldername]