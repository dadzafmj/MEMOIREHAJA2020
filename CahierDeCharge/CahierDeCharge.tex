\documentclass[11pt]{article}
\usepackage[utf8]{inputenc}
\usepackage[frenchb]{babel}
\usepackage[T1]{fontenc}
\usepackage{lmodern}
\usepackage{amsfonts}
\usepackage[left=2cm,right=2cm,top=1.5cm,bottom=2cm]{geometry}
\usepackage{fourier}
\usepackage{tabularx}
\usepackage{multirow}
\usepackage{graphicx}
\usepackage{setspace}

\sloppy
\hyphenpenalty=10000

%% some invisible "struts" to help define the structures and row heights.
\newcommand{\aevstrut}{\rule{0pt}{2.9ex}}
\newcommand{\aehstrut}{\rule{0.45em}{0pt}}
\newcolumntype{L}[1]{>{\raggedright\let\newline\\\arraybackslash\hspace{0pt}}m{#1}}
\newcolumntype{C}[1]{>{\centering\let\newline\\\arraybackslash\hspace{0pt}}m{#1}}
\newcolumntype{R}[1]{>{\raggedleft\let\newline\\\arraybackslash\hspace{0pt}}m{#1}}
%% set up and width for the tabularx environment to expand and fit to.
%\setlength{\headheight}{5cm}

\begin{document}
	
	\thispagestyle{empty}
	%header%
	\begin{minipage}[b]{\textwidth}
		\renewcommand{\arraystretch}{1.29}%
		\begin{tabular}{C{0.21\textwidth}C{0.79\textwidth}}
			\multirow{5}{*}				
			{\includegraphics[width=0.21\textwidth]{logo_stic2.png}} 
			&\textbf{\textsc{{\'E}cole Sup{\'e}rieure Polytechnique d'Antsiranana}}\\
			&\textbf{\textsc{Mention Master STIC}}\\
			&B.P. O 201 - ANTSIRANANA - MADAGASCAR\\
			&\textbf{T{\'e}l. : }+261 (0)32 76 395 40 $-$ \textbf{Courriel : }mentionsticespa@gmail.com\\
		\end{tabular}
	\end{minipage}
	\hrule
	\begin{flushright}
		\og {\textit{Ma{\^i}triser aujourd'hui la technologie de demain}} \fg\\
	\end{flushright}
	
	\begin{center}
		\textbf{\normalsize Projet et Mémoire de fin d'études $-$ A.U : 2018/2019}\\[1.2em]
		
		%Titre du sujet%
		\textbf{\Large{Titre : Conception et  réalisation du logiciel de gestion de la Polyclinique universitaire NEXT}}\\ 
	\end{center}
	
	%Contexte%
	\subsection*{Contexte}
	
		Dans le cadre de son programme de partenariat publique privé, la polyclinique Universitaire Next s'est mise en accord de partenariat avec l'état Malagasy pour la prise en charge des fonctionnaires d'état, depuis juin 2019. Elle souhaite mettre en place un logiciel permettant de gérer les patients ainsi que les activités au sein de l’hôpital. Parmi ces activités sont compris, la réception des patients, gestion de la facturation, gestion de l'hospitalisation et la gestion de la pharmacie. \medskip
		
		Après avoir effectué un état de lieux du logiciel existant, en accord avec l’administration et techniciens, la conception d’un nouveau logiciel s’impose, pour gérer ces fonctionnaires d’état.
	
	%Objectif%
	\subsection*{Objectif}

	L’objectif de ce logiciel est de numériser les données, d’automatiser les activités au sein de la Polyclinique. 
	%Travaux demandes%
	\section*{Travaux demandés}
		
		\subsection*{Cas d'utilisation :}
		Pour l'administration de la polyclinique :
			\begin{itemize}
				\item 	Elaboration de la statistique des patients (Nombre des patients par date, par service...)
				\item 	Facturation des patients (Payement, payement du reste à payer, annulation, modification,...)
				\item 	Gestion des prestations diverses
				\item Bilan régulier
				
			\end{itemize}
			
			
			Pour la réception :
			\begin{itemize}
				\item 	Enregistrement des patients
				\item 	Triage des patients
				\item  	Enregistrement des sorties des patients
				\item  	Facturation
				\item   Saisie des données médicales des patients
			\end{itemize}
			
			Pour l'administration du logiciel :
		
		\begin{itemize}
			\item Gestion des utilisateurs
			\item Gestion des fonctionnalités du logiciel
		\end{itemize}
				
			
	
		%Documentation%
	
		%Encadreur%
		\subsection*{Encadreur(s)}
			\begin{itemize}
				\item	RAKOTOARISOA Jean Claude, Dr Ingénieur
				\item	RAMANAN'HAJA Hery Tina, Ingénieur
				\item	Dr Luigi BELLINI, Professeur
				\item	RAZAIARIMALALA Nirina Claudia, Chef de service Administration
			\end{itemize}
		
		\subsection{Lieu de travail}
		\begin{itemize}
			\item	Laboratoire d'Informatique Appliquée et de Mathématique, UNA
			\item	Polyclinique Universitaire Next
		\end{itemize}
		\subsection{Etudiant réalisateur}
		 HAJALALAINA Fara Marie José
		 
	
\end{document}