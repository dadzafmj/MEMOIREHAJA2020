\documentclass[11pt]{article}
\usepackage[utf8]{inputenc}
\usepackage[frenchb]{babel}
\usepackage[T1]{fontenc}
\usepackage{lmodern}
\usepackage{amsfonts}
\usepackage[left=2cm,right=2cm,top=1.5cm,bottom=2cm]{geometry}
\usepackage{fourier}
\usepackage{tabularx}
\usepackage{multirow}
\usepackage{graphicx}
\usepackage{setspace}

\sloppy
\hyphenpenalty=10000

%% some invisible "struts" to help define the structures and row heights.
\newcommand{\aevstrut}{\rule{0pt}{2.9ex}}
\newcommand{\aehstrut}{\rule{0.45em}{0pt}}
\newcolumntype{L}[1]{>{\raggedright\let\newline\\\arraybackslash\hspace{0pt}}m{#1}}
\newcolumntype{C}[1]{>{\centering\let\newline\\\arraybackslash\hspace{0pt}}m{#1}}
\newcolumntype{R}[1]{>{\raggedleft\let\newline\\\arraybackslash\hspace{0pt}}m{#1}}
%% set up and width for the tabularx environment to expand and fit to.
%\setlength{\headheight}{5cm}

\begin{document}
	
	\thispagestyle{empty}
	%header%
	\begin{minipage}[b]{\textwidth}
		\renewcommand{\arraystretch}{1.29}%
		\begin{tabular}{C{0.21\textwidth}C{0.79\textwidth}}
			\multirow{5}{*}				
			{\includegraphics[width=0.21\textwidth]{logo_stic2.png}} 
			&\textbf{\textsc{{\'E}cole Sup{\'e}rieure Polytechnique d'Antsiranana}}\\
			&\textbf{\textsc{Mention Master STIC}}\\
			&B.P. O 201 - ANTSIRANANA - MADAGASCAR\\
			&\textbf{T{\'e}l. : }+261 (0)32 76 395 40 $-$ \textbf{Courriel : }mentionsticespa@gmail.com\\
		\end{tabular}
	\end{minipage}
	\hrule
	\begin{flushright}
		\og {\textit{Ma{\^i}triser aujourd'hui la technologie de demain}} \fg\\
	\end{flushright}
	
	\begin{center}
		\textbf{\normalsize Projet et Mémoire de fin d'études $-$ A.U : 2016/2017}\\[1.2em]
		
		%Titre du sujet%
		\textbf{\Large{Titre : Développement de SaaS « Group4SuccessManagement »}}\\ 
	\end{center}
	
	%Contexte%
	\subsection*{Contexte}
	Avec ou sans basculement vers le système LMD, toutefois le système d'information étudiants, des établissements d'enseignement supérieur, ne répond plus aux besoins du modèle d'activité existante.
	D'où l'idée de développement d'un SaaS (Software as a Service ou logiciel en tant que service) dénommé « Group4Success » qui est destiné pour : alléger les efforts de gestion de personnel, soutenir la croissance de l'établissement, stimuler la réussite des étudiants et améliorer l'efficacité administrative.
	Comme il s'agit d'un SaaS, l'application et infrastructure sous-jacente sont intégralement hébergées à distance et administrées par les soins de chaque établissement.	
	
	
	%Objectif%
	\subsection*{Objectif}
	C'est de développer et mettre œuvre le SaaS « Group4Success ».
	
	%Travaux demandes%
	\subsection*{Travaux demandés}
		\begin{itemize}
			\item Études théoriques sur les systèmes et services des établissements d'enseignement supérieur (gestion de personnel, administration, scolarité, etc.),
			\item Étude sur le fonctionnement d'un SaaS,
			\item Mise en place de la partie 1 « Alléger les efforts de gestion de personnel »,
			\item Mise en place de la partie 2 « Soutenir la croissance de l'établissement »,
			\item Mise en place de la partie 3 « Stimuler la réussite des étudiants »,
			\item Mise en place de la partie 4 « Améliorer l'efficacité administrative »,
			\item Mise en œuvre du logiciel,
			\item Définition et mise en œuvre des services fournis selon la politique d'un SaaS,
			\item Test et déploiement.
		\end{itemize}

	%Documentation%
	\subsection*{Documentation}
		http ://www.unit4.com/fr/secteurs/education/student-management
		
	%Encadreur%
	\subsection*{Encadreur(s)}
		\begin{itemize}
			\item M. ANDRINIRINIAIMALAZA Fanambinantsoa Philibert,
			\item M. RAMANAN'HAJA Hery Tina,
			\item Encadreur Professionnel.
		\end{itemize}
	
	%Lieu de travail%
	\subsection*{Lieu de travail}
	Laboratoire de Réseaux Informatiques et/ou Laboratoire d'Électronique Industrielle.

\end{document}